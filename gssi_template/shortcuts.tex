\usepackage{tikz-network}
\usepackage{faktor}
\usepackage{comment}

\usepackage{algorithm}% http://ctan.org/pkg/algorithms
\usepackage{algpseudocode}


\usepackage{tikz}
\usepackage{pgfplots}

\usetikzlibrary{arrows.meta}
\usetikzlibrary{backgrounds}
\usepgfplotslibrary{patchplots}
\usepgfplotslibrary{fillbetween}
\pgfplotsset{%
     layers/standard/.define layer set={%
         background,axis background,axis grid,axis ticks,axis lines,axis tick labels,pre main,main,axis descriptions,axis foreground%
     }{
         grid style={/pgfplots/on layer=axis grid},%
         tick style={/pgfplots/on layer=axis ticks},%
         axis line style={/pgfplots/on layer=axis lines},%
         label style={/pgfplots/on layer=axis descriptions},%
         legend style={/pgfplots/on layer=axis descriptions},%
         title style={/pgfplots/on layer=axis descriptions},%
         colorbar style={/pgfplots/on layer=axis descriptions},%
         ticklabel style={/pgfplots/on layer=axis tick labels},%
         axis background@ style={/pgfplots/on layer=axis background},%
         3d box foreground style={/pgfplots/on layer=axis foreground},%
     },
 }




%%%% COMMANDS %%%%%%%%%%%%%%%%%%%%%%%%

\newcommand*{\insidefigure}[3][0.5\columnwidth]{
      \begin{center}
            \begin{minipage}{#1}
                  \centering
                  #2
                  \captionof{figure}{#3}
            \end{minipage}
      \end{center}
}

\newcommand*{\mc}[1]{\mathcal{#1}}
\renewcommand*{\b}[1]{\mathbf{#1}}
\newcommand*\eps{\varepsilon}
\newcommand*{\V}[1]{ \mc V_{#1}(\mc K)}
\newcommand*{\vn}{\varnothing}
\newcommand*{\ord}[1]{\mathrm{ord}\,(#1)}

\usepackage{dsfont}
\newcommand*{\ds}[1]{\mathds{#1}}

\newcommand*{\Lu}[1]{L_{#1}^{\uparrow}}
\newcommand*{\Ld}[1]{L_{#1}^{\downarrow}}

\newcommand*{\wh}[1]{\widehat{#1}}


\renewcommand*{\bar}[1]{ \overline{#1} }


\newcommand{\algname}{\texttt{HeCS}}


\DeclareMathOperator{\im}{im}
\let\span\relax
\DeclareMathOperator{\span}{span}
\DeclareMathOperator{\Sym}{Sym}









\usetikzlibrary{patterns}
\definecolor{bananamania}{rgb}{0.98, 0.91, 0.71}
\definecolor{lavender}{rgb}{0.4470588235294118, 0.5294117647058824, 0.992156862745098}
\definecolor{burntsienna}{rgb}{0.91, 0.45, 0.32}
\definecolor{airforceblue}{rgb}{0.36, 0.54, 0.66}
\definecolor{liberty}{HTML}{5158BB}
\definecolor{junglegreen}{rgb}{0.16, 0.67, 0.53}


\newcommand{\AxisRotator}[1][rotate=0]{%
    \tikz [x=0.15cm,y=0.15cm,line width=.2ex,-stealth,#1] \draw (0,0) arc (-150:150:1 and 1);%
}
\newcommand{\AxisRotatorMirror}[1][rotate=0]{%
    \tikz [x=0.15cm,y=0.15cm,line width=.2ex,-stealth,#1] \draw (0,0) arc (150:-150:1 and 1);%
}
\newcommand{\EVert}[2]{%
\left[\begin{smallmatrix} #1 \\ #2 \end{smallmatrix}\right]
}
\newcommand{\TVert}[3]{%
\left[\begin{smallmatrix} #1 \\ #2 \\ #3 \end{smallmatrix}\right]
}

\usepackage[capitalize, nameinlink]{cleveref}
\crefdefaultlabelformat{\color{liberty}#1#2#3}


\crefname{section}{section}{sections}
\crefname{subsection}{subsection}{subsections}
\Crefname{section}{Section}{Sections}
\Crefname{subsection}{Subsection}{Subsections}
\Crefname{figure}{Figure}{Figures}
\crefformat{equation}{\textup{#2(#1)#3}}
\crefrangeformat{equation}{\textup{#3(#1)#4--#5(#2)#6}}
\crefmultiformat{equation}{\textup{#2(#1)#3}}{ and \textup{#2(#1)#3}}
{, \textup{#2(#1)#3}}{, and \textup{#2(#1)#3}}
\crefrangemultiformat{equation}{\textup{#3(#1)#4--#5(#2)#6}}%
{ and \textup{#3(#1)#4--#5(#2)#6}}{, \textup{#3(#1)#4--#5(#2)#6}}{, and \textup{#3(#1)#4--#5(#2)#6}}

\Crefformat{equation}{#2Equation~\textup{(#1)}#3}
\Crefrangeformat{equation}{Equations~\textup{#3(#1)#4--#5(#2)#6}}
\Crefmultiformat{equation}{Equations~\textup{#2(#1)#3}}{ and \textup{#2(#1)#3}}
{, \textup{#2(#1)#3}}{, and \textup{#2(#1)#3}}
\Crefrangemultiformat{equation}{Equations~\textup{#3(#1)#4--#5(#2)#6}}%
{ and \textup{#3(#1)#4--#5(#2)#6}}{, \textup{#3(#1)#4--#5(#2)#6}}{, and \textup{#3(#1)#4--#5(#2)#6}}

\crefdefaultlabelformat{#2\textup{#1}#3}


% ================================= %
% =========== Colors ============== %
\definecolor{black4}{rgb}{0.11764705882352941, 0.11764705882352941, 0.1803921568627451}
\definecolor{newwhite}{rgb}{0.8509803921568627, 0.8784313725490196, 0.9333333333333333}
\definecolor{newred}{rgb}{0.9098039215686274, 0.6352941176470588, 0.6862745098039216}
\definecolor{newcerulean}{rgb}{0.5882352941176471, 0.803921568627451, 0.984313725490196}
\definecolor{newgreen}{rgb}{0.6705882352941176, 0.9137254901960784, 0.7019607843137254}
\definecolor{newteal}{rgb}{ 0.6039215686274509, 0.8, 0.7686274509803922,}
\definecolor{newpeach}{rgb}{0.9725490196078431, 0.7411764705882353, 0.5882352941176471}
\definecolor{newflam}{rgb}{0.9490196078431372, 0.803921568627451, 0.803921568627451}
\definecolor{persimmon}{HTML}{DE5A02}
\definecolor{jet}{HTML}{333232}
\definecolor{peach}{HTML}{FCDE9C}
\definecolor{cerulean}{HTML}{A7C6DA}
\definecolor{liberty}{HTML}{5158BB}
\definecolor{greensheen}{HTML}{68B0AB}
% =========== Colors ============== %
% ================================= %


%\newcommand{\algname}{\texttt{HeCS}}


\newtheorem{problem}{Problem}

\usepackage{enumitem}

\DeclareMathOperator{\diag}{diag} 

\newenvironment{mt}{\begin{pmatrix}}{\end{pmatrix}}

