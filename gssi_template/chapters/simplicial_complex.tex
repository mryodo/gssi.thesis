\chapter{ Simplicial complex as Higher-order Topology Description } \label{chap:SC}

\lhead{\emph{Simplicial complexes}}



\section{ From graph to higher-order models } \label{sec:higher-order-models}

\todo{graph definition}

\todo{graph examples in real life (2 or 3)}

\todo{motivation for the transition to the higher order models}

\todo{Hypergraphs: definitions and examples}

\todo{Motifs: definitions and examples}

\todo{somehow relate to the tensor models and tractability
simplicial complexes}












\section{ Simplicial Complexes } \label{sec:SCs}


Let \( V = \{ v_1, v_2, \ldots, v_n \} \) be a set of nodes; as discussed above, such set may refer to various interacting entities and agents in the system, e.g.\ neurons, genes, traffic stops, online actors, publication authors, etc. Then: 

\begin{definition}[Simplicial Complex]\label{def:simplicial_complex}
      The collection of subsets \( \mathcal K \) of the nodal set \( V \) is  a (abstract) {SC}\footnote{addition of the word ``abstract'' to the term is more common in the topological setting} if for each subset \( \sigma \in \mc K \), referred as a {simplex}, all its subsets \( \sigma'\), \( \sigma' \subseteq \sigma \), referred as {faces}, enter \( \mc K \) as well, \( \sigma' \in \mc K\).
\end{definition}

A simplex \( \sigma \in \mc K \) on \( k + 1 \) vertices is said to be of the order \( k \), \( \ord \sigma = k \). Let \( \V k \) be a set of all \(k\)-order simplices in \( \mc K \) and \( m_k \) is the cardinality of \( \V k\), \( m_k = | \V k | \); then \( \V 0 \) is the set of nodes in the simplicial complex \( \mc K \), \( \V 1 \) --- the set of edges, \( \V 2 \) --- the set of triangles, or \(3\)-cliques, and so on, with \( \mc K = \{ \V 0, \V 1, \V 2 \ldots \} \). Note that due to the inclusion rule in \Cref{def:simplicial_complex}, the number of non-empty \( \V k \) is finite and, moreover, uninterupted in a sense of the order: if \( \V k = \vn \), then \( \V {k+1} \) is also necessarily empty.
\begin{figure}[hbtp]
      \centering
      \scalebox{1.5}{\begin{tikzpicture}

      \fill [opacity=0.3,liberty]    (0, 0) -- (3, 0) --  (1.5, 2.6) -- cycle;
      %\node at (1.5, 0.9) {\AxisRotator[rotate=-90]};
      %\node at (1.5, 1.2) {\small \color{liberty!50!jet} +1};
      \fill [opacity=0.2,liberty]    (5.0, 1.0) -- (6.5, 2.6) --  (6.4, 1.4) -- cycle;
      %\node at (17.9/3, 5/3) {\AxisRotatorMirror[rotate=0]};
      \fill [opacity=0.6,liberty]    (5.0, 1.0) -- (9.0, -0.4) --  (6.4, 1.4) -- cycle;
      %\node at (6.8, 2/3) {\AxisRotator[rotate=-90]};
      \fill [opacity=0.4,liberty]    (6.5, 2.6) -- (9.0, -0.4) --  (6.4, 1.4) -- cycle;
      %\node at (7.3, 1.2) {\AxisRotatorMirror[rotate=-90]};
      
      


      \Vertex[x=0, y=0,style={color=persimmon}, fontcolor=white, size=0.25, label = 1]{v1}
      %\node[below left=-1pt of v1] {\small \color{persimmon}+1};
      \Vertex[x=3, y=0,style={color=persimmon}, fontcolor=white, size=0.25, label = 2]{v2}
      %\node[below=-1pt of v2] {\small \color{persimmon}-2};
      \Vertex[x=1.5, y=2.6, style={color=persimmon}, fontcolor=white, size=0.25, label = 3]{v3}
     % \node[above=-1pt of v3] {\small \color{persimmon}+0};
      \Vertex[x=5, y=1.0, style={color=persimmon}, fontcolor=white, size=0.25, label = 4]{v4}
    %  \node[above=-1pt of v4] {\small \color{persimmon}-3};
      \Vertex[x=6.5, y=2.6, style={color=persimmon}, fontcolor=white, size=0.25, label = 5]{v5}
   %   \node[above right=-1pt of v5] {\small \color{persimmon}-1};
      \Vertex[x=9, y=-.4, style={color=persimmon}, fontcolor=white, size=0.25, label = 6]{v6}
  %    \node[below=-1pt of v6] {\small \color{persimmon}+0};
      \Vertex[x=6.4, y=1.4, style={color=persimmon}, fontcolor=white, size=0.25, label = 7]{v7}
 %     \node[right=-1pt of v7] {\small \color{persimmon}+2};
      \Vertex[x=11.0, y=2.0, style={color=persimmon}, fontcolor=white, size=0.25, label = 8]{v8}
%      \node[above=-1pt of v8] {\small \color{persimmon}+3};

      \Edge[](v1)(v2)
      \Edge[](v1)(v3)
      \Edge[](v2)(v3)
      \Edge[](v2)(v4)
      \Edge[](v3)(v4)
      \Edge[](v4)(v5)
      \Edge[](v4)(v6)
      \Edge[](v4)(v7)
      \Edge[](v5)(v6)
      \Edge[](v5)(v7)
      \Edge[](v6)(v7)
      \Edge[](v6)(v8)
\end{tikzpicture}}
      \caption{
            Example of a simplicial complex\label{fig:example_SC}}
\end{figure}


\begin{example}[Simplicial Complex]

      Here we provide the following example of the simplicial complex \( \mc K \), \Cref{fig:example_SC}: we denote \(0\)-order simplices (vertices) by orange color, \( 1\)-order simplicies (edges) by black and \( 2\)-order simplices (triangles) by blue, where: 
      \begin{equation}
            \begin{aligned}
                  \V 0 & = \{ 1, 2, 3, 4, 5, 6, 7 \} \\
                  \V 1 & = \{ [1, 2], [1, 3], [1, 4], [2, 3], [2, 4], [2, 5], [3, 4], [3, 5], [4, 6], [4, 7], [5, 6], [6, 7] \} \\
                  \V 2 & = \{ [ 1, 2, 3 ], [1, 2, 4], [1, 3, 4], [2, 3, 5], [4, 6, 7], [5, 6, 7] \}
            \end{aligned}
      \end{equation}
      Note that \( \V 3 = \vn \), so the highest order of simplices in \( \mc K \) is \( 2 \). Additionally, edge \( [2, 3] \), \( [2, 4]\) and \( [3, 4]\) are included in \( \mc K \), but the triangle \( [2, 3, 4]\) is not; this does not violate the inclusion rule. Instead, every edge and every vertices of every triangle in \( \V 2 \) as well as every vertex of every edge in \( \V 1 \) are contained in \( \mc K \) fullfilling the inclusion principle.
\end{example}

\begin{example}[Real Life Simplicial Complex]

      \todo{find a natural example of the simplicial complex with an illustration}
      
\end{example}


Comparing to the general case of the hypergraph described above, it is easy to see that simplicial complex is a special case of a hypergraph where every edge is enclosed with respect to the inclusion. In other words, simplicial complex contains additional structural rigidness which allows to formally describe the topology of \( \mc K \); as a result, one is specifically interested in the formal desctiption of the nested inclusion principle achieved through \emph{boundary operators} defined in the subsections below.


\section{ Hodge's Theory }

Two linear operators \( A \) and \( B \) are said to satisfy Hodge's theory if and only if their composition is a null operator, 
\begin{equation}
      A B = 0
\end{equation}
which is equivalent to \( \im B \subseteq \ker A \).

\begin{definition}\label{def:quotient}
      For a pair of operators \( A \) and \( B \) satisfying Hodge's theory, the \emph{quotient space} \( \mc H \) is defined as follows:
      \begin{equation}
            \mc H = \faktor{ \ker A }{\im B}
      \end{equation}
      where each element of \( \mc H \) is a manifold \( \b x + \im B = \left\{ \b x + \b y \mid \forall \b y \in \im B \right\}\) for \( \b x \in \ker A \). It follows directly from the definition that \( \mc H\) is an abelian group under addition.
\end{definition}

By \Cref{def:quotient}, the quotient space \( \mc H \) is a collection\todo{in a general sense} of \emph{equivalence classes} \( \b x + \im B \). Then, each class \( \b x + \im B = \b x_H + \im B \) for some \( \b x_H \perp \im B \) (both \( \b x, \b x_H \in \ker A \)); indeed, since the orthogonal component \( \b x_H \) (referred as \emph{harmonic representative}) of \( \b x \) with respect to \( \im B\) is unique, the map \( \b x_H \leftrightarrow \b x + \im B \) is bijectional.

\begin{figure}[hbtp]
      \centering
            \newcommand*{\xMin}{-1}%
\newcommand*{\xMax}{2}%
\newcommand*{\yMin}{-2}%
\newcommand*{\yMax}{2}%

\begin{tikzpicture}
      \foreach \i in {\xMin,...,\xMax} {
        \draw [very thin,lightgray] (\i,\yMin-0.2) -- (\i,\yMax+0.2)  node [below] at (\i,\yMin) {};
    }
    \foreach \i in {\yMin,...,\yMax} {
        \draw [very thin,lightgray] (\xMin-0.2,\i) -- (\xMax+0.2,\i) node [left] at (\xMin,\i) {};
    }

    \draw [gray] (\xMin-0.2, 0) -- (\xMax+0.2,0) node [left] at (\xMin,0) {};
    \draw [gray] (0,\yMin-0.2) -- (0,\yMax+0.2)  node [below] at (0,\yMin) {};
    \node at (-1.75, 1.75) { \large \(\ker A \)};

    \draw [black, line width = 3, dashed] (-1, -2) -- (1, 2) node [below] at (0.0, 1.75){ \large \( \im B \) };

    \draw [black, line width = 3] (0.25, -2) -- (2.25, 2) node [below] at (0.0, 1.75){ \large \( \im B \) };

    \draw [liberty, -{latex}, line width = 2] (0, 0) -- (1.15, 0) node [above] at (0.6, 0) {\large \( \b x \)};
    \draw [persimmon, -{latex}, line width = 2] (0, 0) -- (0.9, -.6) node [below] at (0.3, -0.3) {\large \( \b x_H \)};
\end{tikzpicture}
            \caption{Illustration of a harmonic representative for an equivalence class}
\end{figure}

\begin{theorem}[{\cite[Thm 5.3]{limHodgeLaplaciansGraphs2020}}]\label{thm:two_kernels}
      Let \( A \) and \( B \) be linear operators, \( A B = 0 \). Then the homology group \( \mc H \) satisfies:
      \begin{equation}
            \mc H = \faktor{\ker A }{\im B } \cong \ker A \cap \ker B^\top,
      \end{equation}
      where \( \cong \) denotes the isomorphism.
\end{theorem}

\begin{proof}
      One builds the isomorphism through the harmonic representative, as discussed above. It sufficient to note that \( \b x_H \perp \im B \Leftrightarrow \b x_H \in \ker B^\top \) in order to complete the proof.
\end{proof}

\begin{lemma}[{\cite[Thm 5.2]{limHodgeLaplaciansGraphs2020}}]\label{lemma:hodge_kernels}
      Let \( A \) and \( B \) be linear operators, \( A B = 0 \). Then:
      \begin{equation}
            \ker A \cap \ker B^\top = \ker \left( A^\top A + B B^\top \right)
      \end{equation}
      \vspace{-\baselineskip}
\end{lemma}

\begin{proof}
      Note that if \( \b x \in \ker A \cap \ker B^\top \), then \( \b x \in \ker A \) and \( \b x \ker B^\top \), so \( \b x \in \ker \left( A^\top A + B B^\top  \right)\). As a result, \( \ker A \cap \ker B^\top \subset \ker \left( A^\top A + B B^\top  \right)\).

      On the other hand, let \( \b x \in \ker \left( A^\top A + B B^\top \right)\), then
      \begin{equation}
            A^\top A \b x  + B B^\top \b x = 0
      \end{equation}
      Exploiting \( A B = 0 \) and multiplying the equation above by \( B^\top \) and \( A \) one gets the following:
      \begin{equation}
            \begin{aligned}
                  & B^\top B B^\top \b x = 0 \\
                  & A A^\top A \b x = 0 
            \end{aligned}
      \end{equation}
      Note that \( A A^\top A \b x = 0 \Leftrightarrow A^\top A \b x \in \ker A \), but \( A^\top A \b x \in \im A^\top \), so by Fredholm alternative, \( A^\top A \b x = 0\). Finally, for \( A^\top A \b x = 0\):
      \begin{equation}
             A^\top A \b x = 0  \Longrightarrow  \b x^\top A^\top A \b x = 0 \iff \| A \b x \|^2 = 0 \Longrightarrow \b x \in \ker A  
      \end{equation}
      Similarly, for the second equation, \( \b x \in \ker B^\top \) which completes the proof.
\end{proof}

\todo{
      Here we need some words about the transitions.
}


Since \( A B = 0\), \( B^\top A^\top = 0 \) or \( \im A^\top \subset \ker B^\top \). Then, exploiting \( \ds R^n = \ker A \oplus \im A^\top \): 
\begin{equation}
      \begin{aligned}
            \ker B^\top & = \ker B^\top \cap \ds R^n = \ker B^\top \cap \left( \ker A \oplus \im A^\top \right) = \\
            & = \left( \ker A \cap \ker B^\top \right) \oplus \left( \im A^\top \cap \ker B^\top \right)
      \end{aligned}
\end{equation}
Given \Cref{lemma:hodge_kernels}, \( \ker A \cap \ker B^\top = \ker \left(  A^\top A + B B^\top \right) \) and, since \( \im A^\top \subset \ker B^\top \), \( \im A^\top \cap \ker B^\top = \im A^\top \), yeilding the decomposition of  the whole space:
\begin{theorem}[Hodge Decomposition]\label{thm:hodge_decomposition}
      Let \( A \) and \( B \) be linear operators, \( A B = 0 \). Then:
      \begin{equation}
            \ds R^n = \lefteqn{\overbrace{\phantom{\im A^\top \oplus  \ker \left( A^\top A + B B^\top \right)}}^{\ker B^\top}} \im A^\top \oplus
            \underbrace{\ker \left( A^\top A + B B^\top \right) \oplus  \im B}_{\ker A}
      \end{equation}
      \vspace{-\baselineskip}
\end{theorem}



%=================================================%
%                                                 %
%=================================================%


\section{ Boundary and Laplacian Operators }

\subsection{Boundary operators \( B_k \)}

Each simplicial complex \( \mc K \) has a nested structure of simplicies: indeed, if \( \sigma \) is a simplex of order \( k \), \( \sigma \in \V k \), then all of \( (k-1)\)-th order faces forming the boundary of \( \sigma \) also belong to \( \mc K \): for instance, for the triangle \( \{ 1, 2, 3  \} \) all the border edges \( \{ 1, 2\} \), \( \{ 1, 3\}\) and \( \{ 2, 3 \}\) are also in the simplicial complex, \Cref{fig:example_SC}. 

This nested property implies that one can build a formal map from a simplex to its boundary enclosed inside the simplicial complex. 

\begin{definition}[Chain spaces]
      Let \( \mc K \) be a simplicial complex; then the space \( C_k \) of formal sums of simplices from \( \V k \) over real numbers is called \emph{a \( k\)-th chain space}.

      \todo{switch to \( \mc C_k\)}
\end{definition}

\begin{example}    
      \begin{figure}[hbtp]
            \centering
            \begin{tikzpicture}

      \fill [opacity=0.3,liberty]    (0, 0) -- (3, 0) --  (1.5, 2.6) -- cycle;
      \node at (1.5, 0.9) {\AxisRotator[rotate=-90]};
      \node at (1.5, 1.2) {\tiny \color{liberty!50!jet} +1};
      \fill [opacity=0.2,liberty]    (5.0, 1.0) -- (6.5, 2.6) --  (6.4, 1.4) -- cycle;
      \node at (17.9/3, 5/3) {\AxisRotatorMirror[rotate=0]};
      \node at (17.9/3-0.05, 5/3) {\tiny \color{liberty} -1};
      \fill [opacity=0.6,liberty]    (5.0, 1.0) -- (9.0, -0.4) --  (6.4, 1.4) -- cycle;
      \node at (6.8, 2/3) {\AxisRotator[rotate=-90]};
      \node at (6.4, 2/3+0.2) {\tiny \color{liberty!50!jet} +0};
      \fill [opacity=0.4,liberty]    (6.5, 2.6) -- (9.0, -0.4) --  (6.4, 1.4) -- cycle;
      \node at (7.3, 1.2) {\AxisRotatorMirror[rotate=-90]};
      \node at (7.5, 0.9) {\tiny \color{liberty!50!jet} +2};
      


      \Vertex[x=0, y=0,style={color=persimmon}, fontcolor=white, size=0.2, label = 1]{v1}
      \node[below left=-1pt of v1] {\small \color{persimmon}+1};
      \Vertex[x=3, y=0,style={color=persimmon}, fontcolor=white, size=0.2, label = 2]{v2}
      \node[below=-1pt of v2] {\small \color{persimmon}-2};
      \Vertex[x=1.5, y=2.6, style={color=persimmon}, fontcolor=white, size=0.2, label = 3]{v3}
      \node[above=-1pt of v3] {\small \color{persimmon}+0};
      \Vertex[x=5, y=1.0, style={color=persimmon}, fontcolor=white, size=0.2, label = 4]{v4}
      \node[above=-1pt of v4] {\small \color{persimmon}-3};
      \Vertex[x=6.5, y=2.6, style={color=persimmon}, fontcolor=white, size=0.2, label = 5]{v5}
      \node[above right=-1pt of v5] {\small \color{persimmon}-1};
      \Vertex[x=9, y=-.4, style={color=persimmon}, fontcolor=white, size=0.2, label = 6]{v6}
      \node[below=-1pt of v6] {\small \color{persimmon}+0};
      \Vertex[x=6.4, y=1.4, style={color=persimmon}, fontcolor=white, size=0.2, label = 7]{v7}
      \node[right=-1pt of v7] {\small \color{persimmon}+2};
      \Vertex[x=11.0, y=2.0, style={color=persimmon}, fontcolor=white, size=0.2, label = 8]{v8}
      \node[above=-1pt of v8] {\small \color{persimmon}+3};

      \Edge[Direct, label=1, Math](v1)(v2)
      \Edge[Direct, label=-1, Math](v1)(v3)
      \Edge[Direct, label=0, Math](v2)(v3)
      \Edge[Direct, label=2, Math](v2)(v4)
      \Edge[Direct, label=2, Math](v3)(v4)
      \Edge[Direct, label=1, Math](v4)(v5)
      \Edge[Direct, label=-3, Math](v4)(v6)
      \Edge[Direct, label=0, Math](v4)(v7)
      \Edge[Direct, label=0, Math](v5)(v6)
      \Edge[Direct, label=-1, Math](v5)(v7)
      \Edge[Direct, label=2, Math](v6)(v7)
      \Edge[Direct, label=0, Math](v6)(v8)
\end{tikzpicture}
            \caption{Example of chains on the simplicial complex \label{fig:weak_example}}
      \end{figure}

      \todo{add description of the example }
\end{example}

Chain spaces on its own are naturally present in the majority of the network models: \( C_0 \) is a space of states of vertices (e.g. in the dynamical system \( \dot{ \b x } = A \b x \) the evoloving vector \( \b x \in C_0 \)), \( C_1 \) --- is a space of (unrestricted) flows on graphs edges, and so on\todo{refs?}.

Since \( C_k \) is a linear space, versor vectors corresponding to the elements of \( \V k\) is a natural basis of \( C_k \); in order to proceed with the boundary maps, one needs to fix an order of the simplices inside \( \V k \) (alas one fixes the order of vertices in the graph to form an adjacency matrix) as a matter of bookkeeping. Addiitionally, since \( C_k \) is a space over \( \ds R \), chains \( \sigma \) and \( - \sigma \) naturally introduce the notion of \emph{orientation} of the simplex.

\todo{Simplex orientation and order}

\todo{definition of the operators}

\todo{lemma  \( B_k  B_{k+1} = 0 \) in general case}

\todo{ Examples }



\subsection{ Homology group and Hodge Laplacians \( L_k \) }

\subsubsection{Homology group as Quontinent space}

\subsubsection{ Elements of the Hodge decomposition as harmonic/vorticity/potential flow}
      \todo{ Maybe we will need here a quick discussion with connection to the continuous case }

\subsubsection{ Laplacian operators \( L_k \) }

\subsubsection{ Classical Laplacian and its kernel elements }

\subsubsection{ Kernel elements of \( L_k \) }

\todo{Some relation to the continuous case}

\paragraph{ Spreading and balancing as a mechanism of the non-local circulation}
















\section{ Weigthed and Normalised Boundary Operators}

\todo{definition and motivation of the weighting scheme}

Note that, from the definition \( \bar B_k= W_{k-1}^{-1} B_k W_k \) and \eqref{eq:Bk_Bk+1_0}, we immediately have that  \( \bar B_k\bar B_{k+1}=0 \). Thus, the group \( \bar {\mc H}_k = \ker \bar B_k / \im \bar B_{k+1} \) is well defined for any choice of positive weights  \( w_k \) and is isomorphic to \( \ker \bar L_k \). While the homology group may depend on the weights,  we observe below that its dimension does not. Precisely, we have 
\begin{proposition}
 \label{thm:wHomGroup}
     The dimension of the homology groups of \( \mc K \) is not affected by the weights of its \( k \)-simplicies. Precisely, if \( W_k \) are positive diagonal matrices, we have
     \begin{equation}
         \dim \ker \bar B_k = \dim \ker B_k, \quad \dim \ker \bar B_k^\top = \dim \ker B_k^\top, \quad \dim \bar{\mc H}_k = \dim \mc H_k\, .
     \end{equation}
     Moreover, \( \ker B_k = W_k \ker \bar B_k \) and \( \ker B_k^\top = W_{k-1}^{-1} \ker \bar B_k^\top \).
\end{proposition}

\begin{proof}
Since \( W_k \) is an invertible diagonal matrix, 
\begin{equation*}
    \bar B_k \b x = 0 \iff W_{k-1}^{-1}B_k W_k \vec x =0 \iff B_k W_k \vec x =0. 
\end{equation*}
Hence, if \( \vec x \in \ker \bar B_k \), then \( W_k \vec x \in \ker B_k \), and, since \( W_k \) is bijective, \( \dim \ker \bar B_k = \dim \ker B_k \). Similarly, one observes that \( \dim \ker \bar B_k^\top = \dim \ker B_k^\top \).

Moreover, since \( \bar B_k \bar B_{k+1} =0 \), then \( \im \bar B_{k+1} \subseteq \ker \bar B_k \) and \( \im \bar B_k^\top \subseteq \ker \bar B_{k+1}^\top \). This yields \( \ker \bar B_k \cup \ker \bar B_{k+1}^\top = \mathbb{R}^{\mc V_k} = \ker B_k \cup \ker  B_{k+1}^\top\). Thus, for the homology group it holds:
\begin{equation*}
      \begin{aligned}
    \dim \bar {\mc H}_k & = \dim \left( \ker \bar B_k \cap \ker \bar B_{k+1}^\top \right)= \\
    & = \dim \ker \bar B_k + \dim \ker \bar B_{k+1}^\top - \dim \left( \ker \bar B_k \cup \ker \bar B_{k+1}^\top \right) = \\
    & = \dim \ker B_k + \dim \ker  B_{k+1}^\top - \dim \left( \ker  B_k \cup \ker  B_{k+1}^\top \right) = \dim \mc H_k
      \end{aligned}
\end{equation*}
\end{proof}


\todo{normalisation theorem}


