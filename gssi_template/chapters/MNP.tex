\chapter{Topological Stability as MNP}

\section{ General idea of the topological stability }

\subsection{Alternative with a persistent homology}


\subsection{ Transition to the spectral properties }





\section{ 101 on Spectral Matrix Nearness Problems }

\todo{definition}

Generally speaking, for a given matrix \( A \) a \emph{matrix nearness problem} consists of finding the closest possible matrix \( X \) among the admissible set with a number of desired properties. For instance, one may search for the closest (in some metric) symmetric positive/negative definite matrix, unitary matrix or the closest graph Laplacian. 

Motivated by the topological meaning of the \emph{kernel} of Hodge Laplacians \( L_k \), we assume the specific case of \emph{spectral} MNPs: here one aims for the target matrix \( X \) to have a specific spectrum \( \sigma(X) \). For instance in the stability study of the dynamical system \( \dot{\b x} = A \b x \) one can search for the closest Hurwitz matrix such that \( \mathrm{Re} \left[ \lambda_i \right] < 0 \) for all \( \lambda_i \in \sigma(X) \); similarly, assuming given matrix \( A \) is a graph Laplacian, one can search for the closest disconnected graph (so the algebraic connectivity \( \lambda_2 = 0 \)).

Here we recite the optimization framework developed by REFREFREF\todo{fix it} for the class of the spectral matrix nearness problems; one should note, however, that this is by far not the only approach to the task, REFREFREF\todo{also fix it with Nicholas and others, I guess?}.

\subsection{Functional and Gradient Flow}

Let as assume that \( X = A + \Delta \) and intead of searching for \( X \), we search for the perturbation matrix \( \Delta \); additionally, we assume that \( \Omega \) is the admissible set containing all possible perturbations \( \Delta \).


\subsection{ Transition to the gradient flow }
  
  \todo{ Derivative }

\subsection{ Constraint gradient flow }

\subsection{ Sparsity pattern and rank-1 optimizers }

\subsection{ Idea of two level optimization }















\section{Direct approach: failure and discontinuity problems }


\subsection{Principal spectral inheritance}


Before moving on to the next section, we recall here a relatively direct  but important spectral property that connects the spectra of the $k$-th and $(k+1)$-th order Laplacians. 

\begin{theorem}[HOL's spectral inheritance]\label{thm:inherit}
  Let $L_k$ and $L_{k+1}$ be  higher-order Laplacians for the same simplicial complex $\mc K$. Let $\bar L_k=\bar L_k^{down}+\bar L_k^{up}$, where $\bar L_k^{down}=\bar B_k^\top \bar B_k$ and $\bar L_k^{up}=\bar B_{k+1} \bar B_{k+1}^\top$. Then:
  \begin{enumerate}
    \item $\sigma_+(\bar L_k^{up})=\sigma_+(\bar L_{k+1}^{down})$, where $\sigma_+(\cdot)$ denotes the positive part of the spectrum;
    \item if $ 0 \ne \mu \in \sigma_+(\bar L_k^{up}) = \sigma_+(\bar L_{k+1}^{down})$, then the eigenvectors are related as follows:
    \begin{enumerate}
      \item if $\b x$ is and eigenvector for $\bar L_k^{up}$ with the eigenvalue $\mu$, then $\b y = \frac{1}{\sqrt{\mu}} \bar B_{k+1}^\top \b x$ is an eigenvector for $\bar L_{k+1}^{down}$ with the same eigenvalue;
      \item if $\b u$ is and eigenvector for $\bar L_{k+1}^{down}$ with the eigenvalue $\mu$ and $\b u \notin \ker \bar B_{k+1}$, then $\b v = \frac{1}{\sqrt{\mu}} \bar B_{k+1} \b u$ is an eigenvector for $\bar L_{k}^{up}$ with the same eigenvalue;
    \end{enumerate}
    \item for each Laplacian $\bar L_k$: if $\b v \notin \ker \bar L_k^{down}$ is the eigenvector for $\bar L_k^{down}$, then $\b v \in \ker \bar L_{k}^{up}$; vice versa, if $\b u \notin \ker \bar L_k^{up}$ is the eigenvector for $\bar L_k^{up}$, then $\b v \in \ker \bar L_k^{down}$;
    \item consequently, there exist $\mu \in \sigma_+(\bar L_k)$ with an eigenvector $\b u \in \ker \bar L_k^{up}$, and $\nu \in \sigma_+(\bar L_{k+1})$ with an eigenvector $\b u \in \ker \bar L_{k+1}^{down}$, such that:
    $$
    \bar B_k^\top \bar B_k \b v  = \mu \b v, \qquad \bar B_{k+2} \bar B_{k+2}^\top \b u = \nu \b u\, . 
    $$
  \end{enumerate}
\end{theorem}
\begin{proof}
For $(2a)$ it is sufficient to note that $ \bar L_{k+1}^{down} \b y = \bar B_{k+1}^\top \bar B_{k+1} \frac{1}{\sqrt{\mu}} \bar B_{k+1}^\top \b x = \frac{1}{\sqrt{\mu}}\bar B_{k+1}^\top \bar L_k^{up} \b x = \sqrt \mu \bar B_{k+1}^\top \b x  = \mu \b y$. Similarly, for $(2b)$: $\bar L_k^{up} \b v= \bar B_{k+1} \bar B_{k+1}^\top \frac{1}{\sqrt{\mu}} \bar B_{k+1} \b u = \frac{1}{\sqrt{\mu}} \bar B_{k+1} \bar L_{k+1}^{down} \b u = \mu \b v $; joint $2(a)$ and $2(b)$ yield $(1)$. \emph{Hodge decomposition} immediately yields the strict separation of eigenvectors between $\bar L_k^{up}$ and $\bar L_k^{down}$, $(3)$; given $(3)$, all the inherited  eigenvectors from $(2a)$ fall into the $\ker \bar L_{k+1}^{down}$, thus resulting into $(4)$.
\end{proof}
In other words, the variation of the spectrum of the $k$-th Laplacian when moving from one order to the next one works as follows: 
the down-term $\bar L_{k+1}^{down}$ inherits the positive part of the spectrum from the up-term of  $\bar L_k^{up}$; the  eigenvectors corresponding to the inherited positive part of the spectrum lie in the kernel of $\bar L_{k+1}^{up}$; at the same time, the ``new'' up-term $\bar L_{k+1}^{up}$ has a new, non-inherited, part of the positive spectrum (which, in turn, lies in the kernel of the $(k+2)$-th down-term).

In particular, we notice that for $k = 0$, since $B_0=0$ and $\bar L_0=\bar L_0^{up}$, the  theorem yields $\sigma_+ (\bar L_0 ) = \sigma_+ (\bar{L_1}^{down}) \subseteq \sigma_+(\bar L_1)$. In other terms, the positive spectrum of the $\bar L_0$ is inherited by the spectrum of $\bar L_1$ and the remaining (non-inherited) part of $\sigma_+(\bar L_1)$ coincides with $\sigma_+(\bar L_1^{up})$. 
\Cref{fig:thm_spct_ill} provides an  illustration of the statement of  \Cref{thm:inherit} for $k = 0$.
  \begin{figure}[t]
    \centering
    \begin{tikzpicture}
      \node[draw] at (0,0) {0};
      \node[draw] at (0.5,0) {0};
      \node at (1, 0) {$\cdots$};
      \node[draw] at (1.5,0) {0};
      \node[draw, fill=bananamania] at (2.15,0) {\tiny{$\lambda_1$}};
      \node[draw, fill=bananamania] at (2.65,0) {\tiny{$\lambda_2$}};
      \node[draw, fill=burntsienna] at (3.15,0) {\tiny{$\lambda_3$}};
      \node[draw, fill=bananamania] at (3.65,0) {\tiny{$\lambda_4$}};
      \node[draw, fill=burntsienna] at (4.15,0) {\tiny{$\lambda_5$}};
      \node[draw, fill=burntsienna] at (4.65,0) {\tiny{$\lambda_6$}};
      \node[draw, fill=bananamania] at (5.15,0) {\tiny{$\lambda_7$}};
      \node[draw, fill=bananamania] at (5.65,0) {\tiny{$\lambda_8$}};
      \node[draw, fill=burntsienna] at (6.15,0) {\tiny{$\lambda_9$}};
      \node[draw, fill=bananamania] at (6.65,0) {\tiny{$\lambda_{10}$}};
      \node at (8.5, 0) {$\leftarrow \quad \sigma (\bar L_1)$\phantom{$\bar B_1$}};
    
      \node[draw] at (0,-0.7) {0};
      \node[draw] at (0.5,-0.7) {0};
      \node at (1, -0.7) {$\cdots$};
      \node[draw] at (1.5,-0.7) {0};
      \node[draw, fill=bananamania] at (2.15,-0.7) {\tiny{$\lambda_1$}};
      \node[draw, fill=bananamania] at (2.65,-0.7) {\tiny{$\lambda_2$}};
      \node[draw] at (3.15,-0.7) {0};
      \node[draw, fill=bananamania] at (3.65,-0.7) {\tiny{$\lambda_4$}};
      \node[draw] at (4.15,-0.7) {0};
      \node[draw] at (4.65,-0.7) {0};
      \node[draw, fill=bananamania] at (5.15,-0.7) {\tiny{$\lambda_7$}};
      \node[draw, fill=bananamania] at (5.65,-0.7) {\tiny{$\lambda_8$}};
      \node[draw] at (6.15,-0.7) {0};
      \node[draw, fill=bananamania] at (6.65,-0.7) {\tiny{$\lambda_{10}$}};
      \node at (8.5, -0.7) {$\leftarrow \quad \sigma (\bar B_1^T \bar B_1)$};
    
      \node[draw] at (0,-1.4) {0};
      \node[draw] at (0.5,-1.4) {0};
      \node at (1, -1.4) {$\cdots$};
      \node[draw] at (1.5,-1.4) {0};
      \node[draw] at (2.15,-1.4) {0};
      \node[draw] at (2.65,-1.4) {0};
      \node[draw, fill=burntsienna] at (3.15,-1.4) {\tiny{$\lambda_3$}};
      \node[draw] at (3.65,-1.4) {0};
      \node[draw, fill=burntsienna] at (4.15,-1.4) {\tiny{$\lambda_5$}};
      \node[draw, fill=burntsienna] at (4.65,-1.4) {\tiny{$\lambda_6$}};
      \node[draw] at (5.15,-1.4) {0};
      \node[draw] at (5.65,-1.4) {0};
      \node[draw, fill=burntsienna] at (6.15,-1.4) {\tiny{$\lambda_9$}};
      \node[draw] at (6.65,-1.4) {0};
      \node at (8.5, -1.4) {$\leftarrow \quad \sigma (\bar B_2 \bar B_2^T)$};
    
      \draw [
        thick,
        decoration={
            brace,
            mirror,
            raise=0.25cm
        },
        decorate
      ] (-0.25, -1.4) -- (1.75, -1.4) 
    node [pos=0.5,anchor=north,yshift=-0.25cm] {holes}; 
      \draw[pattern=north west lines] (1.75,0.4) rectangle (1.85, -1.8);
      \node[draw, align=center, fill=bananamania] at (1.8,-2.1) {$\mu$};
    \end{tikzpicture}
    \caption{Illustration for the principal spectrum inheritance (\Cref{thm:inherit}) in case $k=0$: spectra of $\bar L_1$, $\bar {\Ld 1}$ and $\bar {\Ld 1}$ are shown. Colors signify the splitting of the spectrum, $\lambda_i>0 \in \sigma(\bar L_1)$ ; all yellow eigenvalues are inherited from $\sigma_+(\bar L_0)$; red eigenvalues belong to the non-inherited part. Dashed barrier $\mu$ signifies the penalization threshold (see the target functional in \Cref{subsec:functional}) preventing homological pollution (see \Cref{subsec:connetedness}). }
    \label{fig:thm_spct_ill}
    \vspace{-10pt}
  \end{figure}


\subsection{ Example with inheritted disconnectedness }



\subsection{ Example with faux edges (different weighting scheme) }





\section{ Functional, derivative and alternating scheme }

\subsection{ Target Functional }


\subsection{ Free gradient calculation }


\subsection{ Constrained gradient }


\subsection{ Alternating scheme }


\subsection{ Implementation }

\subsubsection{ Algorithms }

\subsubsection{ Computation of the first non-zero eigenvalue }

\subsubsection{ Preconditioning in the eigen-phase }







\section{ Benchmarking }

\subsection{ Toy example }


\subsection{ Triangulation }

\todo{Preconditioning of the LS as a way forward}



\subsection{ Cities }


\begin{table}[hbtp]
  \centering
  \begin{NiceTabular}{@{}c!{\qquad}cccc!{\qquad}ccc@{}}
    \toprule
    \Block{2-1}{\bfseries Cities} & \Block{1-3}{\bfseries network} & & & \Block{2-1}{ \( \beta_1 \) } & \Block{1-3}{\bfseries logarithmic weights} \\
    & \( m_0 \) & \( m_1 \) & \( m_2 \) & & time & \( \eps \) & \( p \) \\
    \midrule
    \Block{2-1}{Bologna} & \Block{2-1}{60} & \Block{2-1}{175} & \Block{2-1}{171} & \Block{2-1}{2} &  \( 2.43 \)s & \( 0.65 \) & \( 0.003 \)\\
     &  &  &  & & \Block{1-3}{\small  \textbf{{[}11, 47{]}} ($4^{th}$ smallest) } \\[0.1cm]
     \dashedline
     \noalign{\vskip 0.1cm}
     \Block{2-1}{Anaheim} & \Block{2-1}{38} & \Block{2-1}{159} & \Block{2-1}{221} & \Block{2-1}{1} &  \( 5.39 \)s & \( 0.57 \) & \(0.003\)\\
     &  &  &  &  & \Block{1-3}{\small \textbf{{[}10, 29{]}} ($11^{th}$ smallest)} \\[0.1cm]
     \dashedline
     \noalign{\vskip 0.1cm}
     \Block{2-1}{Berlin-Tiergarten} & \Block{2-1}{26} & \Block{2-1}{63} & \Block{2-1}{55} & \Block{2-1}{0} & \(2.46\)s & \(1.18\) & \(0.015\) \\
     & &  & & & \Block{1-3}{\small \textbf{{[}6, 16{]}} ($20^{th}$ smallest)} \\[0.1cm]
     \dashedline
     \noalign{\vskip 0.1cm}
     \Block{2-1}{Berlin-Mitte} & \Block{2-1}{98} & \Block{2-1}{456} & \Block{2-1}{900} & \Block{2-1}{1} &  \( 127 \)s & \(0.887\) & \(0.0016\)\\
     & & & & & \Block{1-3}{\small \textbf{{[}57, 87{]}} ($6^{th}$), \textbf{{[}58, 87{]}}, ($17^{th}$) } \\
    \bottomrule
    \end{NiceTabular}

    \caption{
      Topological instability of the transportation networks: filtered zone networks with the corresponding perturbation norm \( \eps \) and its percentile among \( w_1(\cdot) \) profile. For each simplicial complex the number of nodes, edges and triangles in \( \mc V_2(\mc K) \) are provided alongside the initial number of holes \( \beta_1 \). The results of the algorithm consist of the perturbation norm, \( \eps \), computation time, and approximate percentile \( p \).\label{tab:bologna}
    }
  
\end{table}

































