\section{Matrix nearness problems}



\subsection{ 101 on MNP }

Generally speaking, for a given matrix \( A \) a \emph{matrix nearness problem} consists of finding the closest possible matrix \( X \) among the admissible set with a number of desired properties.
























%Let \( A \) be a square matrix, \( A \ in \ds C^{n \times n }\); \todo{do we need that?} let \( \lambda ( A )\) be a chosen eigenvalue of the matrix \( A \). For instance, \( \lambda \) can be the eigenvalue with highest/lowest magnitude or real part on the complex plane.

%Generally speaking, \emph{a spectral matrix nearness problem} focuses on finding a matrix \( X \) closest to \( A \) such that the spectrum \( \sigma ( X ) \) satisfies some property. For instance, one can search for the closest matrix such that the rightmost eigenvalue has a zero real part on the complex plane (in the stability study of the continuous dynamical systems) or all the eigenvalues fit on the unit disk (in the stability study of the discrete dynamical systems), etc.

%Let \( \Delta = X - A \); then one aims to find a perturbation \( \Delta \) with a minimal norm, \( \min \| \Delta \| \). Additionally, 