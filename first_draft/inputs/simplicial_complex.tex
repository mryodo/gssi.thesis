\section{ From Graphs to Simplicial Complex }

%\subsection{ Higher-order Models in Networks }

\subsection{ Simplicial Complexes }

Let \( V = \{ v_1, v_2, \ldots, v_n \} \) be a set of nodes; as discussed above, such set may refer to various interacting entities and agents in the system, e.g.\ neurons, genes, traffic stops, online actors, publication authors, etc. Let \( \sigma \) be a \emph{simplex}\mnl{for the purposes of the current work, a subset of \( V \)} on \( (k+1) \)-node subset of \( V \); then we refer to it as simplex of order \( k \) and all its \( (k-1) \)-order subsimplices as \emph{faces}. Then: 

\begin{definition}[Simplicial Complex]\label{def:simplicial_complex}
      The collection of subsets \( \mathcal K \) of the nodal set \( \{ v_1, v_2, \ldots, v_n \} \) is  a \emph{simplicial complex} if each element \( \sigma \in \mc K \), referred as a \emph{simplex}, enters \( \mc K \) with all its faces.
\end{definition}

Let \( \V k \) be a set of all \(k\)-order simplices in \( \mc K \) and \( m_k \) is the caridnality of \( \V k\), \( m_k = | \V k | \); then \( \V 0 \) is the set of nodes in the simplicial complex \( \mc K \), \( \V 1 \) --- the set of edges, \( \V 2 \) --- the set of triangles, or \(3\)-cliques, and so on, with \( \mc K = \{ \V 0, \V 1, \V 2 \ldots \} \). Note that due to the inclusion rule in \Cref{def:simplicial_complex}, the number of non-empty \( \V k \) is finite and, moreover, uninterupted in a sense of the order: if \( \V k = \vn \), then \( \V {k+1} \) is also necessarily empty.
\en{
      \insidefigure[0.3\columnwidth]{ \begin{tikzpicture}

      \fill [opacity=0.3,liberty]    (0, 0) -- (3, 0) --  (1.5, 2.6) -- cycle;
      %\node at (1.5, 0.9) {\AxisRotator[rotate=-90]};
      %\node at (1.5, 1.2) {\small \color{liberty!50!jet} +1};
      \fill [opacity=0.2,liberty]    (5.0, 1.0) -- (6.5, 2.6) --  (6.4, 1.4) -- cycle;
      %\node at (17.9/3, 5/3) {\AxisRotatorMirror[rotate=0]};
      \fill [opacity=0.6,liberty]    (5.0, 1.0) -- (9.0, -0.4) --  (6.4, 1.4) -- cycle;
      %\node at (6.8, 2/3) {\AxisRotator[rotate=-90]};
      \fill [opacity=0.4,liberty]    (6.5, 2.6) -- (9.0, -0.4) --  (6.4, 1.4) -- cycle;
      %\node at (7.3, 1.2) {\AxisRotatorMirror[rotate=-90]};
      
      


      \Vertex[x=0, y=0,style={color=persimmon}, fontcolor=white, size=0.25, label = 1]{v1}
      %\node[below left=-1pt of v1] {\small \color{persimmon}+1};
      \Vertex[x=3, y=0,style={color=persimmon}, fontcolor=white, size=0.25, label = 2]{v2}
      %\node[below=-1pt of v2] {\small \color{persimmon}-2};
      \Vertex[x=1.5, y=2.6, style={color=persimmon}, fontcolor=white, size=0.25, label = 3]{v3}
     % \node[above=-1pt of v3] {\small \color{persimmon}+0};
      \Vertex[x=5, y=1.0, style={color=persimmon}, fontcolor=white, size=0.25, label = 4]{v4}
    %  \node[above=-1pt of v4] {\small \color{persimmon}-3};
      \Vertex[x=6.5, y=2.6, style={color=persimmon}, fontcolor=white, size=0.25, label = 5]{v5}
   %   \node[above right=-1pt of v5] {\small \color{persimmon}-1};
      \Vertex[x=9, y=-.4, style={color=persimmon}, fontcolor=white, size=0.25, label = 6]{v6}
  %    \node[below=-1pt of v6] {\small \color{persimmon}+0};
      \Vertex[x=6.4, y=1.4, style={color=persimmon}, fontcolor=white, size=0.25, label = 7]{v7}
 %     \node[right=-1pt of v7] {\small \color{persimmon}+2};
      \Vertex[x=11.0, y=2.0, style={color=persimmon}, fontcolor=white, size=0.25, label = 8]{v8}
%      \node[above=-1pt of v8] {\small \color{persimmon}+3};

      \Edge[](v1)(v2)
      \Edge[](v1)(v3)
      \Edge[](v2)(v3)
      \Edge[](v2)(v4)
      \Edge[](v3)(v4)
      \Edge[](v4)(v5)
      \Edge[](v4)(v6)
      \Edge[](v4)(v7)
      \Edge[](v5)(v6)
      \Edge[](v5)(v7)
      \Edge[](v6)(v7)
      \Edge[](v6)(v8)
\end{tikzpicture} }{
            Example of a simplicial complex}}

\begin{example}[Simplicial Complex]

      123
      
\end{example}