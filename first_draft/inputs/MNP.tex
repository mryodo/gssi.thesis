\section{Matrix nearness problems}



\subsection{ 101 on MNP }

Generally speaking, for a given matrix \( A \) a \emph{matrix nearness problem} consists of finding the closest possible matrix \( X \) among the admissible set with a number of desired properties. For instance, one may search for the closest (in some metric) symmetric positive/negative definite matrix, unitary matrix or the closest graph Laplacian. 

Motivated by the topological meaning of the \emph{kernel} of Hodge Laplacians \( L_k \), we assume the specific case of \emph{spectral} MNPs: here one aims for the target matrix \( X \) to have a specific spectrum \( \sigma(X) \). For instance in the stability study of the dynamical system \( \dot{\b x} = A \b x \) one can search for the closest Hurwitz matrix such that \( \mathrm{Re} \left[ \lambda_i \right] < 0 \) for all \( \lambda_i \in \sigma(X) \); similarly, assuming given matrix \( A \) is a graph Laplacian, one can search for the closest disconnected graph (so the algebraic connectivity \( \lambda_2 = 0 \)).

Here we recite the optimization framework developed by REFREFREF\todo{fix it} for the class of the spectral matrix nearness problems; one should note, however, that this is by far not the only approach to the task, REFREFREF\todo{also fix it with Nicholas and others, I guess?}.

\paragraph{Functional and Gradient Flow}

Let as assume that \( X = A + \Delta \) and intead of searching for \( X \), we search for the perturbation matrix \( \Delta \); additionally, we assume that \( \Omega \) is the admissible set containing all possible perturbations \( \Delta \).

























%Let \( A \) be a square matrix, \( A \ in \ds C^{n \times n }\); \todo{do we need that?} let \( \lambda ( A )\) be a chosen eigenvalue of the matrix \( A \). For instance, \( \lambda \) can be the eigenvalue with highest/lowest magnitude or real part on the complex plane.

%Generally speaking, \emph{a spectral matrix nearness problem} focuses on finding a matrix \( X \) closest to \( A \) such that the spectrum \( \sigma ( X ) \) satisfies some property. For instance, one can search for the closest matrix such that the rightmost eigenvalue has a zero real part on the complex plane (in the stability study of the continuous dynamical systems) or all the eigenvalues fit on the unit disk (in the stability study of the discrete dynamical systems), etc.

%Let \( \Delta = X - A \); then one aims to find a perturbation \( \Delta \) with a minimal norm, \( \min \| \Delta \| \). Additionally, 