\section{ From Graphs to Simplicial Complex }

%\subsection{ Higher-order Models in Networks }

\subsection{ Simplicial Complexes }

Let \( V = \{ v_1, v_2, \ldots, v_n \} \) be a set of nodes in the system; as discussed above such set may correspond to various entities and agents in the system. Let \( \sigma \) be a \emph{simplex}\mnl{for the purposes of the current work, a subset of \( V \)} on \( (k+1) \)-node subset of \( V \); then we refer to it as simplex of order \( k \) and all its \( (k-1) \)-order subsimplices as \emph{faces}. Then: 

\begin{definition}[Simplicial Complex]
      \label{def:simplicial_complex}
      The collection of subsets \( \mathcal K \) of the nodal set \( \{ v_1, v_2, \ldots, v_n \} \) is  a \emph{simplicial complex} if each element \( \sigma \in \mc K \), referred as a \emph{simplex}, enters \( \mc K \) with all its faces.
\end{definition}

Let \( \V k \) be a set of all \(k\)-order simplices in \( \mc K \) and \( m_k \) is the caridnality of \( \V k\), \( m_k = | \V k | \); then \( \mc K = \{ \V 0, \V 1, \V 2 \ldots \} \). Note that due to the inclusion rule in \Cref{def:simplicial_complex}, the number of non-empty \( \V k \) is finite and, moreover, uninterupted in a sense of the order: if \( \V k = \vn \), then \( \V {k+1} \) is also necessarily empty.

